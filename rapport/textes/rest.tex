\chapter{Choix des protocoles et standards pour les échanges de données}

Pour cette partie backoffice de l'application de carte de cantine, nous avions
deux contraintes à respecter. La première était d'utiliser un protocole d'échange
 de données compatible avec Android. La seconde était de choisir un format de données
 facilement exploitable.

 Comme vous allez le voir maintenant, nous avons choisi le couple REST/JSON pour cette
 tâche.

\section{Le REpresentational State Transfert ou REST}

La première chose à préciser est que REST n'est pas un protocole en lui même, mais
plutôt une architecture utilisant le protocole HTTP. L'architecture est de donc de type
CRUD (Create, Retrieve, Update, Delete), mais nous n'utiliserons que le GET de HTTP,
qui correspond à la récupération, à la lecture d'une ou plusieurs entrées en base.

De par son universalité, le HTTP ne pose aucun problème quant à la compatibilité
avec la partie Android du frontoffice. Pour accéder aux données, chaque ressource doit
avoir une URI (Uniform Ressource Identifier), à laquelle on fait un appel via une
des méthodes HTTP en fonction de l'opération CRUD à réaliser. Côté Laravel, cette
gestion des routes et méthodes est gérée nativement, ce qui permet une mise en place
rapide et efficace de cette architecture.

\section{Le format de données}

Mais le choix du protocole et de l'architecture n'est pas suffisant pour définir
la communication entre notre serveur et la partie frontoffice. En effet, nous devons
aussi choisir sous quel format ces données sont échangées. Plusieurs possibilités
se sont offertes à nous.

\subsection{Le XML}

Format connu et le plus utilisé, le XML est intégré par défaut dans de nombreux
langages. Il permet d'exploiter les données même si l'on ne connait pas leur structure.
Il a en effet vocation à être un langage de présentation.

L'inconvénient cependant est qu'il est plutôt compliqué à générer et transformer
en objet.

\subsection{Le JSON}

JSON est le format qui monte en ce moment et qui vient de JavaScript où il est géré
nativement. Il n'est pas intégré de base dans la plupart des langages mais dispose
de nombreuses implémentations qui permettent de le gérer facilement. Il est plus
léger que le XML et permet d'être lisible plus facilement par l'homme.

L'inconvénient, c'est qu'il faut s'entendre sur le format de la ressource car celle-ci
n'est pas décrite et structurée explicitement comme dans le xml.

L'application frontend ne requetant que de simples adresses NFC au serveur, notre
choix c'est porté sur le JSON car les données échangées sont simples (nom, prénom,
adresse NFC) et que ce format de données est géré par android et Laravel.
