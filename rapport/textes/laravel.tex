\chapter{Utilisation du framework}

Comme énoncé dans la première partie, nous avons fait le choix d'utiliser un framework nommé Laravel. Ce framework est assez récent et a pour objectif de faciliter le développement. Son slogan est "PHP That Doesn't Hurt. Code Happy \& Enjoy The Fresh Air."

\section{Routes}

Dans la partie précédente, nous avons les différentes routes nécessaire à la mise en place d'un CRUD. Laravel implémente cela facilement via une façade nommée tout simplement \verb|Route|.

Notre application contient une seule route, définie de la manière suivante dans le fichier app/route.php :
\begin{minted}{php}
Route::get('users/batch', 'UsersController@batch');
\end{minted}

Tout d'abord, nous appelons la méthode \verb|get| parce que dans une optique CRUD, nous cherchons à obtenir des informations. Cette méthode prend en premier paramètre l'URI associé à la route et en deuxième paramètre une chaîne de caractère  formatée comme \verb|ControllerObject@method|. Dans notre cas, lorsque l'utilisateur fait une requête GET sur l'URI \verb|users/batch|, Laravel va appeler la méthode \verb|batch| sur l'objet \verb|UsersController|.

\subsection{Qu'est-ce qu'une façade}

Une façade est un classe spécifique qui permet d'appeler des méthodes publiques d'une instance d'un objet via des appels statiques sur la façade.

Par exemple, lorsque j'appelle une méthode statique sur la façade \verb|Route|, Laravel va en réalité appeler une méthode publique d'une instance d'une classe appelée Illuminate\\Routing\\Router stockée dans l'application.

Vous pouvez en savoir plus sur les façades Laravel dans la documentation http://laravel.com/docs/façades

\section{Le modèle}

Laravel utilise son propre ORM appelé Eloquent. Ce dernier nécessite la création d'une classe par ressource de l'application. Notre application a donc un modèle nommé \verb|User| situé dans le répertoire \verb|app/models|. Ce dernier hérite du modèle Eloquent qui fournit toutes les méthodes d'accès à la base de données telles que \verb|where|, \verb|find|, \verb|save|...

Dans notre modèle, nous devons définir plusieurs attributs :
\begin{itemize}
   \item \verb|$table| : cet attribut permet de définir la table associée au modèle. Même si Eloquent est assez intelligent pour définir le pluriel du nom de l'objet (User => users), pour des raisons de lisibilité, nous l'avons redéfini.
   \item \verb|$fillable| : afin de répondre au problème de \verb|MassAssigment|, Eloquent demande de fournir un tableau contenant les attributs du modèle qu'il sera possible de sauvegarder directement via la méthode \verb|create| afin d'éviter qu'un utilisateur mal intentionné insère dans la base de données des champs non voulus. (plus d'informations sur http://laravel.com/docs/eloquent\#mass-assignment)
\end{itemize}
\section{Les contrôleurs}

Toujours dans une optique CRUD, Laravel nous encourage à créer un contrôleur pour chaque ressource. Notre application ne contient qu'une seule ressource pour le moment appelée \verb|User|, notre contrôleur (par convention) aura donc le nom de \verb|UsersController| et sera situé dans le répertoire \verb|app/controllers|.

Cet objet aura une seule méthode, la méthode \verb|batch| définie dans les routes comme la méthode à appeler.

\section{La méthode batch}

Dans notre unique méthode, nous pouvons définir cinq phases.
  * Récupération des données fournies en GET ;
  * Test des données et renvoi d'une erreur en cas d'absence d'identifiants ;
  * Récupération des données en BDD via un répertoire (repository) ;
  * Validation de ces données et information en cas de données manquantes ;
  * Retour de l'ensemble du résultat si tout s'est bien passé.

\section{Utilisation d'un répertoire}

\subsection{Pourquoi utiliser un répertoire}

Dans le cas d'une application simple comme la notre, il aurait été parfaitement correct d'appeler les méthodes de l'ORM directement dans le contrôleur comme simplifié ci-dessous :
\begin{minted}{php}
public function batch()
{
  return User::whereIn('serial', Input::get('data'));
}
\end{minted}

Mais dans le cas d'une application plus complexe, il est préférable de dissocier ses contrôleurs de l'ORM afin d'améliorer la maintenabilité de l'application et de permettre l'utilisation d'un autre système de stockage.

\subsection{Utilisation d'un répertoire dans notre projet}

Afin de montrer les possibilités du framework Laravel ainsi que de permettre aux futurs étudiants allant travailler sur l'application de partir sur de bonnes bases, nous avons décidé d'implémenter un répertoire pour les utilisateurs.

Ce repertoire se présente sous la forme d'une classe nommée \verb|DatabaseUserRepository| proposant une seule méthode : \verb|getBySerialBatch|. Cette méthode prend en paramètre un tableau contenant les différents numéros de carte NFC et retourne une collection d'objets \verb|User| via la méthode \verb|whereIn| de l'ORM Eloquent.

Maintenant, afin de découpler nos contrôleurs de l'ORM, nous devons utiliser notre répertoire. Pour cela, Laravel nous propose deux solutions :
  * Instancier l'objet normalement dans la méthode ;
  * Utiliser l'outil d'injection de dépendance de Laravel.

\subsection{Outil d'injection de dépendance}

Nous n'instançons jamais notre contrôleur, Laravel s'en charge à notre place lorsque la route enregistrée est demandée. Pour instancier une classe, Laravel utilise un outil très puissant de résolution de dépendance grâce à l'introspection de PHP. Avant de créer l'objet nécessaire, elle détermine les classes requises par le constructeur.

Si le constructeur nécessite un objet \verb|DatabaseUserRepository| comme dans notre cas, Laravel va l'instancier et le fournir au constructeur du contrôleur \verb|UsersController|.
\begin{minted}{php}
public function __construct(DatabaseUserRepository $userRepository)
{
  $this->userRepository = $userRepository;
}
\end{minted}

De cette manière, nous avons accès au répertoire dans notre contrôleur sans effort.

\subsection{"Code to an interface"}

La méthode présentée dans le paragraphe précédent présente l'avantage d'être simple à implémenter mais présente aussi un gros défaut : notre contrôleur est toujours lié à l'ORM via l'objet \verb|DatabaseUserRepository|.

En réalité, notre contrôleur n'a pas besoin de savoir quel type de répertoire il utilise, l'unique besoin est une fonction appelée \verb|getBySerialBatch| prenant en paramètre un tableau d'identifiants NFC et retournant une collection d'utilisateurs. Ce qui est décrit ici est le principe même d'une interface. Nous avons donc créé une interface appelée \verb|UserRepository| et fournissant la signature de méthode requise. Bien évidement, notre répertoire \verb|DatabaseUserRepository| doit maintenant implémenter cette interface.

Nous pouvons donc changer le pré-requis du constructeur de notre contrôleur par l'interface \verb|UserRepository|.
\begin{minted}{php}
public function __construct(UserRepository $userRepository)
{
  $this->userRepository = $userRepository;
}
\end{minted}

Mais que va donc faire Laravel lorsqu'il va rencontrer l'interface. En effet, si le framework cherche à résoudre la dépendance, il va se heurter au problème qu'il est impossible d'instancier une interface. Nous avons donc besoin de définir un lien entre notre interface et notre implémentation.

\subsection{Service provider}

Les \verb|services provider| de Laravel sont de simples objets fournissant de la configuration pour l'application. Ils sont définis dans le fichier \verb|app/config/app.php|.

Dans notre cas, nous avons créé un simple service provider \verb|LeoCardServiceProvider| qui effectue le lien de l'interface vers l'implémentation dans sa méthode \verb|register|.
\begin{minted}{php}
public function register()
{
  $this->app->bind('UserRepository', 'DatabaseUserRepository');
}
\end{minted}

\subsection{Conclusion}

Cette approche par répertoire et par interface nous a permis de créer une application très solide et sans grande dépendance. Par exemple, si nous souhaitons changer d'implémentation, remplacer par exemple la base de données par un système de fichier, il suffit de définir un nouveau repertoire \verb|FileUserRepository| implémentant l'interface \verb|UserRepository| et de compléter les méthodes nécessaires. Ensuite, en changant le lien dans le service provider nous obtenons une nouvelle implémentation sans jamais avoir à ouvrir nos contrôleurs ce qui aurait pu être une source de bugs lors du changement.

La principale limitation de ce système provient du language PHP, qui ne permet pas de spécifier des types de retour. La nouvelle implémentation doit donc se référer à la documentation et suivre à la lettre les instructions de retour spécifiées dans l'interface même si ces dernières ne sont pas obligatoire au niveau du language. Ce problème peut être contourné via l'utilisation de la machine virtuelle \verb|HHVM| de Facebook et du language \verb|Hack|, compatible avec PHP et introduisant un typage statique.
